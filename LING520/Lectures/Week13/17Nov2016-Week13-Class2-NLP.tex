\documentclass{beamer}
\usepackage[utf8]{inputenc}
\usepackage{graphicx}

\hypersetup{
    colorlinks,%
    citecolor=blue,%
    filecolor=blue,%
    linkcolor=blue,%
    urlcolor=blue 
    %urlcolor=mygreylink     % can put red here to better visualize the links
}

\author[Sowmya Vajjala]{Instructor: Sowmya Vajjala}

\title[LING 520]{LING 520: Computational Analysis of English}
\subtitle{Semester: FALL '16}

\date{17 November 2016}

\institute{Iowa State University, USA}
%%%%%%%%%%%%%%%%%%%%%%%%%%%

\begin{document}

\begin{frame}\titlepage
\end{frame}

\begin{frame}
\frametitle{Class Outline}
\begin{itemize}
\item Assignment 5 Discussion
\item Announcements
\item NLP for CALL: overview of topics and research
\item Update on final project status
\end{itemize}
\end{frame}

\begin{frame} %20min
\frametitle{}
\Large Assignment 5 Discussion
\end{frame}

\begin{frame} %10min
\frametitle{Few announcements}
\begin{itemize}
\item Grade improvement option: Do any of the following problems in Problem sets: 2.1-2.10, 3.1-3.6, 5.3, 7.5-7.10. Each problem carries 1 mark, and you can get a maximum of 7 points improvement. Send your code as a zip file, in an email, along with a pdf file listing what problems you solved from what problem set. Deadline: 8th December. \pause
\item Class on 29th: Evgeny will give a talk about CyWrite. Read the following paper before attending the class:
\\ \textit{Feng, H. H., Saricaoglu, A., \& Chukharev-Hudilainen, E. (2016). Automated Error Detection for Developing Grammar Proficiency of ESL Learners. CALICO Journal, 33, 49-70.}
\\ (paper is uploaded on Blackboard) \pause
\end{itemize}
\end{frame}

\begin{frame}
\frametitle{}
\Large NLP for CALL - summaries of readings
\begin{itemize}
\item Id\'{e}e on Meurers (2012)
\item Liberato and Evan on Burstein (2009)
\item Lena on Chapelle and Chung (2010)
\end{itemize}
\end{frame}


\begin{frame}
\frametitle{Where is NLP useful for CALL?}
There are two broad uses of NLP in the context of CALL.
\begin{itemize}
\item Analysing the language of learners (feedback, assessment, annotations etc)
\item Analysing the non-learner language, for learners (suggesting materials, creating tests etc) 
\end{itemize}
\end{frame}

\begin{frame}
\frametitle{}
\Large NLP for analyzing learner language
\end{frame}

\begin{frame}
\frametitle{NLP for CALL: analyzing learner language}
\begin{itemize}
\item Automated scoring of learner essays (e.g., in GRE exam)
\item in language tutoring systems (e.g., TAGARELA, described in language and computers textbook) to create activities, give feedback, model a learner's error patterns etc.
\item in offering on-the-fly feedback on writing (e.g., CyWrite, Grammarly, etc)
\item in analysis and annotation of learner corpora (e.g., MERLIN \url{http://merlin-platform.eu/})
\item (often handled seperately): Analysing and assessing spoken responses of language learners
\end{itemize}
\end{frame}

\begin{frame}
\frametitle{NLP beyond "language form" but within "learner language"}
\begin{itemize}
\item Intelligent tutoring systems to teach science, psychology etc (holding dialogues with users, analysing their responses)
\item systems like RWT which provide writing feedback, but more about content, and not about grammatical form
\item scoring short responses of learners to questions about something they read
\end{itemize}
\end{frame}

\begin{frame}
\frametitle{On using NLP for analyzing learner language}
\framesubtitle{From Meurers (2012), page 2:}
\textit{"However, there is an important difference in the goal of the NLP use in an ILTS compared to
that in other NLP domains. NLP is made robust to gloss over errors and unexpected aspects of
the system input with the goal of producing some result, such as a syntactic analysis returned
by a parser, or a translation provided by a machine translation system. The traditional goal of
the NLP in an ILTS, on the other hand, is to identify the characteristics of learner language
and in which way the learner responses diverge from the expected targets in order to provide
feedback to the learner. So errors here are the goal of the abstraction performed by the NLP,
not something to be glossed over by robustness of processing."}
\end{frame}

\begin{frame}
\frametitle{On writers aids and learner language}
\framesubtitle{From Meurers (2012), page 2} \small
\textit{"Writer’s aids such as the standard spell and grammar checkers (Dickinson, 2006) share the
ILTS focus on identifying errors, but they rely on assumptions about typical errors made
by native speakers which do not carry over to language learners. For example, Rimrott
\& Heift (2008) observe that “in contrast to most misspellings by native writers, many L2
misspellings are multiple-edit errors and are thus not corrected by a spell checker designed
for native writers.” Tschichold (1999) also points out that traditional writer’s aids are not
necessarily helpful for language learners since learners need more scaffolding than a list of
alternatives from which to chose. Writer’s aids tools targeting language learners, such as the
ESL Assistant (Gamon et al., 2009), therefore provide more feedback and, e.g., concordance
views of alternatives to support the language learner in understanding the alternatives and
choosing the right. The goal of writer’s aids is to support the second language user in writing
a functional, well-formed text, not to support them in acquiring the language as is the goal of
an ILTS."}
\end{frame}

\begin{frame}
\frametitle{}
\Large NLP for analyzing language FOR learners
\end{frame}

\begin{frame}
\frametitle{NLP for CALL: analyzing language FOR learners}
\begin{itemize}
\item Selecting appropriate texts for learners (e.g., SourceRater/TextEvaluator by ETS: \url{https://texteval-pilot.ets.org/TextEvaluator/})
\item Modifying text (e.g., simplifying) for learners (e.g., Text Adoptor research prototype by ETS described in Burstein, 2009)
\item Language Muse system -linguistically focused instructional authoring (ETS)
\item Input enhancement - highlighting certain grammatical parts of a webpage to make learners notice (e.g., WERTI and FLAIRS, from University of Tuebingen)
\item Creating multiple choice questions (or other forms of questions)
\end{itemize}
\end{frame}

\begin{frame}
\frametitle{Useful resources to find readings}
\begin{itemize}
\item CALL, ReCALL, CALICO etc journals in ALT community
\item Educational Data Mining, AI in Education conferences and journals
\item Building Educational Applications using NLP series of events
\item International Speech Communication Association (ISCA) Special Interest Group: Speech and Language Technologies in Education (SLaTE) 
\item International Journal of Learner Corpus Research (not always computational)
\item NLP4CALL workshop series in Sweden
\item Natural Language Processing Techniques for Educational Application (NLP-TEA) organized by Chinese NLP community
\item Shared tasks on grammatical error correction in ACL conferences
\end{itemize}
... and so on
\end{frame}

\begin{frame}
\frametitle{Next Week}
\begin{itemize}
\item NLP for CALL continued
\item Evgeny's talk on tuesday. Read the article mentioned earlier in the class.
\item We will have discussions and group presentations on all those those report writing questions in your assignments too.
\end{itemize}
\end{frame}

\begin{frame}
\frametitle{Updates on final projects}
\begin{itemize}
\item who is doing what?
\item If you end up finishing your final project during the break, please volunteer to give a short presentation in the class in the last week. Do not consider this as extra work. Be proud of your project and be willing to show it off and get feedback.
\item reminder about grading for the final project: 2.5 marks for initial report, 2.5 marks for final report. 10 marks for implementation. 5 marks for oral exam. 
\end{itemize}
\end{frame}

\begin{frame}
\frametitle{}
\Large Happy Thanksgiving! Have a good break!
\end{frame}

\end{document}

