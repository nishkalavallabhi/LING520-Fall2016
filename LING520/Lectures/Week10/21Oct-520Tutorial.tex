\documentclass{beamer}
\usepackage[utf8]{inputenc}
\usepackage{graphicx}

\hypersetup{
    colorlinks,%
    citecolor=blue,%
    filecolor=blue,%
    linkcolor=blue,%
    urlcolor=blue 
    %urlcolor=mygreylink     % can put red here to better visualize the links
}

\author[Sowmya Vajjala]{Instructor: Sowmya Vajjala}

\title[LING 520]{LING 520: Computational Analysis of English}
\subtitle{Semester: FALL '16}

\date{21 October 2016}

\institute{Iowa State University, USA}
%%%%%%%%%%%%%%%%%%%%%%%%%%%

\begin{document}

\begin{frame}\titlepage
\end{frame}

\begin{frame}
\frametitle{Class Outline}
\begin{itemize}
\item Practice with what to do before writing a program
\item Reading a dictionary from a text file - My question to you
%\item Probability and Programming - discussion
%\item Programming exercises
\end{itemize}
\end{frame}

\begin{frame}
\frametitle{What to do before writing a program}
\begin{enumerate}
\item Choosing the right data structure for a task
\item Writing loops appropriately
\end{enumerate}
\end{frame} %20min

\begin{frame}
\frametitle{What to do before writing a program - Instructions}
\begin{itemize}
\item The idea of next few slides is to make you think through and analyze what you need to write a program that suits the problem description.
\item You should not write programs. You should only try to work through the problem scenarios and come with some pseudo-code
\item Pseudo-code: where you describe what are your variables, what kind of objects are they (string, list, dictionary etc), what are the loops you should have etc.
\item Working in groups is encouraged. 
\item Let us have 15-20 minutes for each question, and one person should come and explain their solution on whiteboard after that.
\end{itemize}
\pause Note: This may look "useless" and "theoretical", but you can never "apply" anything without knowing some "theory". 
\end{frame}

\begin{frame}
\frametitle{When to use what data structure - 1}
\framesubtitle{Morse Code}
\begin{itemize}
\item Task: a program to convert a sentence it into morse code.
\item Questions to address before starting to program:
\begin{enumerate}
\item How will you take input from user?
\item What information do you need to convert it to morse code? 
\item How will that information be stored internally?
\item How will the output be displayed to the user?
\item Once you have answers, write the pseudo code - describing step by step how the program should look.
\end{enumerate}
\end{itemize}
\end{frame}

\begin{frame}
\frametitle{When to use what data structure - 2}
\framesubtitle{Ngram counting}
\begin{itemize}
\item Let us say all books written by Author 1 are collected and stored in a single file author1.txt, and all books written by Author 2 in author2.txt.
\item There are two problems: a) list the most frequent 100 bigrams in each of these authors' writings. b) compare how many bigrams overlap between these authors.
\item What sort of information will you compute while writing this program?
\item Once you have the answer, write the pseudo code - describing step by step how the program should look.
\end{itemize}
\end{frame}

\begin{frame}
\frametitle{When to use what data structure - 3}
\framesubtitle{POS Tagging}
\begin{itemize}
\item Let us say you have access to a large newspaper corpus, and your goal is to answer the following questions:
\begin{enumerate}
\item Problem 1: Find out which nouns are more commonly used in singular form and which are used in plural form.
\item Problem 2: Which tags are nouns most commonly found after? What do these tags represent?
\item Problem 3: How many words are ambiguous, in the sense that they appear with at least two tags?
\end{enumerate}
\item For all these problems, how will you design your solutions? What information do you need? What kind of data structures will you use to store data? 
\item Once you have answers, write the pseudo code
\end{itemize}
Note: based on Exercises 15--18 in Chapter 5 of NLTK book
\end{frame}

\begin{frame}
\frametitle{Writing Loops Appropriately -1}
\framesubtitle{Duplicate items in lists}
\begin{itemize}
\item Problem: Write one program that takes a "list of numbers" as input from user, and prints out a list that shows numbers that repeat in this list. 
\item If I input [1,2,33,2,33,11], my program should output [2,33].
\item How will you solve this problem? What are the issues you may face? 
\item Write your pseudo code.\pause
\item In the code sample I will show now, there are two functions. Predict what they will print. (DuplicateItems.py)
\end{itemize}
\end{frame}

\begin{frame}
\frametitle{Writing Loops Appropriately -2}
\framesubtitle{}
\begin{itemize}
\item Problem 1: Program should take a string as input, and print every third character in the string, if it is not a 'a'. \pause
\item Problem 2:  Program should take a string as input, and replace each character with the character that follows it in English alphabet (i.e., if input is "Peter", it should become "Qfufs")
\item Again, think through and prepare a pseudo code. 
\end{itemize}
\pause ForIf.py
\end{frame}

\begin{frame}
\frametitle{}
calling a function from one program in another - skipping this as Stephanie showed it on Tuesday. We also discussed this in 516, if you remember. 
\end{frame}

\end{document}
\begin{frame}
\frametitle{Reading a saved dictionary}
\begin{itemize}
\item Question: How do you store a dictionary on to your hard disk? How will you access this dictionary in another program? How many people successfully did this? How?\pause
\item Here is one way of doing it: dictionary-creation.py, dictionary-use.py
\end{itemize}
\end{frame}

\begin{frame}
\frametitle{Probability and Programming}
\begin{itemize}
\item Katrin Erk's code, line by line + more.
\item Note: This is the best and most comprehensible code on this topic I found so far. You are more than welcome to find something easier than this and share with the class if you think I am wasting your time by repeatedly talking about this. 
\end{itemize}
%Go through the code line by line.
\end{frame}

\begin{frame}
\frametitle{Programming exercise - 1}
\begin{itemize}
\item Take any .txt version file from gutenberg.org, written by your favorite author (in English!)
\item Read that file into your python code, and do the following using NLTK:
\begin{enumerate}
\item Split the file into sentences.
\item Print the following: number of sentences in the file, average sentence length (in number of words), average word length (in number of characters), number of unique words, number of unique stems.
\item Note 1: Once sentence splitting is done, you can ignore punctuation markers for the rest of the calculation.
\item Note 2: You can use any stemmer you want.
\end{enumerate}
\end{itemize}
\end{frame}

\begin{frame}
\frametitle{Programming exercise - 2}
\begin{itemize}
\item For the same file from last slide, do the following:
\item Read that file into your python code, and do the following using NLTK:
\begin{enumerate}
\item Split the file into sentences.
\item For each sentence, print its POS tagged version as a string of tags. E.g., if you have "It is a sentence" as your sentence, and your NLTK tagger outputs the tags for this sentence as a list (or tuple or whatever), you should print the tag as a string. E.g., PRP VBZ DT NN (not as [PRP, VBZ, DT, NN] or as [(It, PRP), (is, VBZ) .. ... ])
\end{enumerate}
\end{itemize}
\end{frame}






