\documentclass{beamer}
\usepackage[utf8]{inputenc}

\hypersetup{
    colorlinks,%
    citecolor=blue,%
    filecolor=blue,%
    linkcolor=blue,%
    urlcolor=blue 
    %urlcolor=mygreylink     % can put red here to better visualize the links
}

\author[Sowmya Vajjala]{Instructor: Sowmya Vajjala}

\title[LING 520]{LING 520: Computational Analysis of English}
\subtitle{Semester: FALL '16}

\date{30 August 2016}

\institute{Iowa State University, USA}
%%%%%%%%%%%%%%%%%%%%%%%%%%%

\begin{document}

\begin{frame}\titlepage
\end{frame}

\begin{frame}
\frametitle{Class outline}
\begin{itemize}
\item Programming project demos %10 min
\item Week 1 - Recap %10 min, Ask if there are any questions about the syllabus, course policies etc.
\item Review of programming concepts %30 min
\item Practice: Problem Set 1 %30 min
\end{itemize}
\end{frame}

\begin{frame}
\begin{center}
\Large Programming project demos
\end{center}
\end{frame}

\begin{frame}
\frametitle{}
Any questions about Week 1 content, course syllabus/policies etc? \\ 
.. or about the coursera lectures? 
\end{frame}

\begin{frame}
\frametitle{}
American Association of Corpus Linguistics conference is happening in ISU in a few weeks. Try to register and attend! 
\end{frame}

\begin{frame}
\frametitle{}
\begin{center}
\Large Review of programming concepts
\end{center}
\end{frame}

\begin{frame}
\frametitle{Basic building blocks in one slide}
\begin{itemize}
\item Defining variables, writing different expressions to change the variables in the program.
\pause \item Writing conditional statements (if condition, do this; else do that) \pause
\item Writing loops (for, while) to iteratively repeat some set of expressions until some condition is met.
\pause \item Manipulating different data structures (strings, Arrays/lists, Dictionaries/Hashes etc)
\pause \item Reading and Writing data into files on your hard disk.
\end{itemize}
\end{frame}

\begin{frame}
\frametitle{Some practice with your programming language syntax}
\end{frame}

\begin{frame}
\frametitle{Basic Building Blocks: Variables and Expressions}
Do the following in a program or in the interactive console.
\begin{itemize}
\item Define three variables: a, b, c and assign values: 3, 4, and "a" respectively.
\item print the output of expressions: a + b, a + c
\item execute the expression a = b and print the value of a after that.
\item execute: a = 5, and then, print the value of (a+b)*(a-b)
\end{itemize}
\end{frame}


\begin{frame}
\frametitle{Basic Building Blocks: Conditional statements}
Do the following in a program.
\begin{itemize}
\item Write a program that accepts some input from the user.
\item If the user enters "Hello", the program prints "Hello". For any other input, the program prints "Politeness first". 
\end{itemize}
\end{frame}

\begin{frame}
\frametitle{Basic Building Blocks: Loops}
\begin{itemize}
\item Write a program that prints numbers from 1 to 20. 
\end{itemize}
\end{frame}

\begin{frame}
\frametitle{Basic Building Blocks: Data Structures}
\begin{itemize}
\item Write a program that takes a sentence, and splits at whitespace.
\end{itemize}
\end{frame}


\begin{frame}
\frametitle{Basic Building Blocks: Files}
\begin{itemize}
\item Write a program that reads in a sample .txt file, prints the file contents, and appends one additional line to the file. 
\end{itemize}
\end{frame}


\begin{frame}
\frametitle{Some non-syntax based exercises}
\end{frame}

\begin{frame}[fragile]
\frametitle{What is this below pseudo-code doing?}
%Palindrome
\begin{verbatim}
(Let us say str is a string)
length = len(str) #number of characters in the string
i = 0
while i<length:
    if str[i] != str[length-1-i]
          return False
    else:
          continue
    i = i+1
return True
\end{verbatim}
\pause What is returned if str has a value "exception"? 
\end{frame}


\begin{frame}[fragile]
\frametitle{What is this below pseudo-code doing?}
%Factorial
\begin{verbatim}
function/subroutine magic(n):
if n <= 0:
   return 0
else if n == 1:
   return 1
else:
   return magic(n-1) * n
\end{verbatim}	
\pause What is returned if n = 5? 6? 
\end{frame}

\begin{frame}[fragile]
\frametitle{What is this below pseudo-code doing?}
%Stemming: Porter stemmer rules
\begin{verbatim}
function change(word):
  if word.endswith("ational"):
      replace "ational" with "ate"
  else if word.endswith("tional"):
      replace "tional" with "tion"
  return word
\end{verbatim}
\pause What will this return for "relational" and "conditional"?
\end{frame}

\begin{frame}[fragile]
\frametitle{What happens if I change the order of the if statements?}
\begin{verbatim}
function change(word):
  if word.endswith("tional"):
      replace "tional" with "tion"
  else if word.endswith("ational"):
      replace "ational" with "ate"
  return word
\end{verbatim}
\pause What will this return for "relational" and "conditional"?
\\ Note: Your rules in Question 1 of Assignment 1 should look like this.
\end{frame}

\begin{frame}[fragile]
\frametitle{What is this below pseudocode doing?}
\begin{verbatim}
content = read(some_text_file)
freq_data = {} #Empty dictionary. 
words_list = split_into_words(content)
for word in words_list:
   length = len(word)
   if (len > 5):
        if word in freq_data:
            freq_data[word] = freq_data[word] + 1
        else:
            freq_data[word] = 1
display freq_data
\end{verbatim}
What will freq\_data contain, if we program this logic? 
\end{frame}

\begin{frame}
\frametitle{Errors while programming}
\begin{itemize}
\item Syntax errors: You should know it is a syntax error by seeing the error messages (missing colons, parantheses etc)
\pause \item Logic errors: Slightly difficult to find out, because you may not get an error message. You may only get a wrong answer. One solution: Debug your code. Put print statements in the middle, wherever you have questions.
\pause \item Other errors: such as division by zero, inputting a decimal number when a string is expected, etc. - use try-catch/try-except blocks in your programming to handle such cases.
\end{itemize}
\end{frame}

\begin{frame}[fragile]
\frametitle{Start working on Problem Set 1}
... we will briefly discuss Problem set 1 on thursday, and move on to other practice exercises. 

\\ Something that can be useful: pythontutor.com lets you visualize your python code line by line, showing what variables get what values etc. 
\end{frame}

\end{document}

%For thursday: NLTK, Padre installation, some practice exercises etc.
http://perlmaven.com/how-to-install-a-perl-module-from-cpan


