\documentclass{beamer}
\usepackage[utf8]{inputenc}
\usepackage{graphicx}

\hypersetup{
    colorlinks,%
    citecolor=blue,%
    filecolor=blue,%
    linkcolor=blue,%
    urlcolor=blue 
    %urlcolor=mygreylink     % can put red here to better visualize the links
}

\author[Sowmya Vajjala]{Instructor: Sowmya Vajjala}

\title[LING 520]{LING 520: Computational Analysis of English}
\subtitle{Semester: FALL '16}

\date{1 December 2016}

\institute{Iowa State University, USA}
%%%%%%%%%%%%%%%%%%%%%%%%%%%

\begin{document}

\begin{frame}\titlepage
\end{frame}

\begin{frame}
\frametitle{Class Outline}
\begin{itemize}
\item Final project presentation by Ivana and Ziwei, and Hyunwoo
\item Discussion about usage of NLP tools with learner language  %30min
\end{itemize}
\end{frame}

\begin{frame} %15min
\frametitle{}
\Large Final project presentations
\end{frame}

\begin{frame}
\frametitle{Discussion on usage of NLP tools with learner language }
\begin{itemize}
\item Have a look at Discussion.pdf uploaded in this week's content folder on BlackBoard. 
\item It contains the non-programming questions about how existing NLP tools work with learner language, which were a part of your Assignments 3--5.
\item Your task: form into teams of 2-3 people, discuss your answers, and come out with a set of observations as a team, for each of those assignments. 
\item Spend some time discussing, and each team should present their conclusions after some time.
\end{itemize}
\end{frame}

\begin{frame}
\frametitle{Next Week (last week!) - Readings}
\begin{itemize}
\item Case study of real-life usage of NLP for language and  content teachers \\
\textit{Burstein  et.al.,(2014). From Teacher Professional Development to the Classroom: How NLP Technology Can Enhance Teachers’ Linguistic Awareness to Support Curriculum Development for English Language Learners. Journal of Educational Computing Research, 51(1): 119-144}
\\ (pdf on BlackBoard) 
\item Case study of usage of NLP for writing support for learners \\
\textit{Burstein et.al., (2016). A Left Turn: Automated Feedback \& Activity Generation for Student Writers. To appear in the Proceedings of the 3rd Language Teaching, Language \& Technology Workshop, co-located with Interspeech, San Francisco, CA.} \\ (pdf on Blackboard)
\end{itemize}
Note: Two volunteers needed to present these.
\end{frame}

\begin{frame} %20min
\frametitle{Next Week - Plan}
\begin{itemize}
\item Discussion on readings (tuesday)
\item Revision and wrapping up
\item Thursday: Final project presentations by Liberato, Sonca, Stephanie and any other enthusiasts. 
\item Reminder: Final project code and report due by 9th December
\item oral exams: 9-13 December. Fix a time by sending me an email. Time required: 10-15 min. My preferred timing: 1-4 pm on any day during that week. 
\end{itemize}
\end{frame}

\end{document}

\begin{frame}
\frametitle{Discussion on Burstein et.al. (2014)}
\textit{Burstein et.al,. From Teacher Professional Development to the Classroom: How NLP Technology Can Enhance Teachers’ Linguistic Awareness to Support Curriculum Development for English Language Learners. Journal of Educational Computing Research, 51(1): 119-144}
\\ (pdf on BlackBoard) 
\end{frame}

\begin{frame}
\frametitle{Discussion on Burstein et.al. (2016)}
\textit{Burstein et.al., (2016). A Left Turn: Automated Feedback \& Activity Generation for Student Writers. To appear in the Proceedings of the 3rd Language Teaching, Language \& Technology Workshop, co-located with Interspeech, San Francisco, CA.} \\ (pdf on Blackboard)
\end{frame}


\begin{frame} %15min
\frametitle{}
\Large NLP for CALL - discussion
\end{frame}

\begin{frame}
\frametitle{Where is NLP useful for CALL?}
There are two broad uses of NLP in the context of CALL.
\begin{itemize}
\item Analysing the language of learners (feedback, assessment, annotations etc)
\item Analysing the non-learner language, for learners (suggesting materials, creating tests etc) 
\end{itemize}
\end{frame}

\begin{frame}
\frametitle{}
\Large NLP for analyzing learner language
\end{frame}

\begin{frame}
\frametitle{NLP for CALL: analyzing learner language}
\begin{itemize}
\item Automated scoring of learner essays (e.g., in GRE exam)
\item in language tutoring systems (e.g., TAGARELA, described in language and computers textbook) to create activities, give feedback, model a learner's error patterns etc.
\item in offering on-the-fly feedback on writing (e.g., CyWrite, Grammarly, etc)
\item in analysis and annotation of learner corpora (e.g., MERLIN \url{http://merlin-platform.eu/})
\item (often handled seperately): Analysing and assessing spoken responses of language learners
\end{itemize}
\end{frame}

\begin{frame}
\frametitle{NLP beyond "language form" but within "learner language"}
\begin{itemize}
\item Intelligent tutoring systems to teach science, psychology etc (holding dialogues with users, analysing their responses)
\item systems like RWT which provide writing feedback, but more about content, and not about grammatical form
\item scoring short responses of learners to questions about something they read
\end{itemize}
\end{frame}

\begin{frame}
\frametitle{On using NLP for analyzing learner language}
\framesubtitle{From Meurers (2012), page 2:}
\textit{"However, there is an important difference in the goal of the NLP use in an ILTS compared to
that in other NLP domains. NLP is made robust to gloss over errors and unexpected aspects of
the system input with the goal of producing some result, such as a syntactic analysis returned
by a parser, or a translation provided by a machine translation system. The traditional goal of
the NLP in an ILTS, on the other hand, is to identify the characteristics of learner language
and in which way the learner responses diverge from the expected targets in order to provide
feedback to the learner. So errors here are the goal of the abstraction performed by the NLP,
not something to be glossed over by robustness of processing."}
\end{frame}

\begin{frame}
\frametitle{On writers aids and learner language}
\framesubtitle{From Meurers (2012), page 2} \small
\textit{"Writer’s aids such as the standard spell and grammar checkers (Dickinson, 2006) share the
ILTS focus on identifying errors, but they rely on assumptions about typical errors made
by native speakers which do not carry over to language learners. For example, Rimrott
\& Heift (2008) observe that “in contrast to most misspellings by native writers, many L2
misspellings are multiple-edit errors and are thus not corrected by a spell checker designed
for native writers.” Tschichold (1999) also points out that traditional writer’s aids are not
necessarily helpful for language learners since learners need more scaffolding than a list of
alternatives from which to chose. Writer’s aids tools targeting language learners, such as the
ESL Assistant (Gamon et al., 2009), therefore provide more feedback and, e.g., concordance
views of alternatives to support the language learner in understanding the alternatives and
choosing the right. The goal of writer’s aids is to support the second language user in writing
a functional, well-formed text, not to support them in acquiring the language as is the goal of
an ILTS."}
\end{frame}

\begin{frame}
\frametitle{}
\Large NLP for analyzing language FOR learners
\end{frame}

\begin{frame}
\frametitle{NLP for CALL: analyzing language FOR learners}
\begin{itemize}
\item Selecting appropriate texts for learners (e.g., SourceRater/TextEvaluator by ETS: \url{https://texteval-pilot.ets.org/TextEvaluator/})
\item Modifying text (e.g., simplifying) for learners (e.g., Text Adoptor research prototype by ETS described in Burstein, 2009)
\item Language Muse system -linguistically focused instructional authoring (ETS)
\item Input enhancement - highlighting certain grammatical parts of a webpage to make learners notice (e.g., WERTI and FLAIRS, from University of Tuebingen)
\item Creating multiple choice questions (or other forms of questions)
\end{itemize}
\end{frame}

%One exercise. ADD. approx 20min - continuation from yesterday.
%Reading: comparison of parsers. EMNLP paper??

\end{document}

