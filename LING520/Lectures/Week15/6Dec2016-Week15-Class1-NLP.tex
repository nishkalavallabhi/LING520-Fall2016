\documentclass{beamer}
\usepackage[utf8]{inputenc}
\usepackage{graphicx}

\hypersetup{
    colorlinks,%
    citecolor=blue,%
    filecolor=blue,%
    linkcolor=blue,%
    urlcolor=blue 
    %urlcolor=mygreylink     % can put red here to better visualize the links
}

\author[Sowmya Vajjala]{Instructor: Sowmya Vajjala}

\title[LING 520]{LING 520: Computational Analysis of English}
\subtitle{Semester: FALL '16}

\date{6 December 2016}

\institute{Iowa State University, USA}
%%%%%%%%%%%%%%%%%%%%%%%%%%%

\begin{document}

\begin{frame}\titlepage
\end{frame}

\begin{frame}
\frametitle{Class Outline}
\begin{itemize}
\item Announcement: Please complete course evaluations! Less than half of the class did it so far! 
\item Project presentations (Stephanie and Evan)
\item Discussion about usage of NLP tools with learner language  %30min
%\item (if there is time) Discussion about two articles on user studies in NLP for CALL.
\end{itemize}
\end{frame}

\begin{frame}
\frametitle{Discussion on usage of NLP tools with learner language }
\begin{itemize}
\item Have a look at Discussion.pdf uploaded in this week's content folder on BlackBoard. 
\item It contains the non-programming questions about how existing NLP tools work with learner language, which were a part of your Assignments 3--5.
\item Your task: form into teams of 2-3 people, discuss your answers, and come out with a set of observations as a team, for each of those assignments. 
\item Spend some time discussing, and each team should present their conclusions after some time.
\end{itemize}
\end{frame}

\begin{frame} %20min
\frametitle{Next class}
\begin{itemize}
\item Project presentations (Liberato, Sonca, Zaha)
\item Revision and wrapping up
\item oral exams for others: 9-13 December. Fix a time by sending me an email. Time required: 10-15 min. My preferred timing: 1-4 pm on any day during that week. 
\end{itemize}
\end{frame}

\end{document}

\begin{frame}
\frametitle{(if there is time) more Discussion}
\begin{itemize}
\item Case study of real-life usage of NLP for language and  content teachers \\
\textit{Burstein  et.al.,(2014). From Teacher Professional Development to the Classroom: How NLP Technology Can Enhance Teachers’ Linguistic Awareness to Support Curriculum Development for English Language Learners. Journal of Educational Computing Research, 51(1): 119-144}
\\ (pdf on BlackBoard) 
\item Case study of usage of NLP for writing support for learners \\
\textit{Burstein et.al., (2016). A Left Turn: Automated Feedback \& Activity Generation for Student Writers. To appear in the Proceedings of the 3rd Language Teaching, Language \& Technology Workshop, co-located with Interspeech, San Francisco, CA.} \\ (pdf on Blackboard)
\end{itemize}
Note: Two volunteers needed to present these.
\end{frame}
