\documentclass[11pt,a4paper]{article}

% some more symbols
\usepackage{textcomp}

\usepackage[utf8]{inputenc}

\usepackage{natbib,multicol}
\bibpunct[, ]{(}{)}{;}{a}{}{,}

\setlength{\parindent}{0cm}
\setlength{\parskip}{1ex}
\addtolength{\oddsidemargin}{-7ex}
\addtolength{\evensidemargin}{-7ex}
\addtolength{\textwidth}{14ex}
\addtolength{\topmargin}{-2\baselineskip}
\addtolength{\textheight}{4\baselineskip}

% Ensure that we see the local urls that are in the bib file:
%\newcommand{\localurl}[1]{ OSU local copy: \url{file:#1}}

% \begin{htmlonly}
% \renewcommand{\href}[2]{\htmladdnormallink{#1}{#2}}
% \end{htmlonly}

%begin{latexonly}
%\renewcommand{\mylink}[2]{\href{#1}{#2}}

\usepackage{url}
%\usepackage[colorlinks,citecolor=blue,pdfpagemode=FullScreen]{hyperref}

%\urlstyle{rm}
%\def\UrlSpecials{\do\~{\mbox{\~{}}}\do_{\_}\do\%{}}

%end{latexonly}

\usepackage[breaklinks,colorlinks,filecolor=blue,linkcolor=blue,urlcolor=blue,citecolor=red]{hyperref}

% for regular paper output:
%\hypersetup{}

\usepackage{url}

\begin{document}

\begin{center}
  \textbf{Fall Semester 2016 \\ Iowa State University\\[3ex]
  {\Large LING 520 - Computational Analysis of English}\\[3ex]
  Course Handbook
}
\end{center}

\bigskip
%\newpage
\textbf{\large Instructor:}
  Sowmya Vajjala
  \begin{itemize}\vspace*{-.4\baselineskip}\itemsep-.4ex
  \item \textit{Office:} 331 Ross Hall
  \item \textit{Email:} sowmya@iastate.edu
\end{itemize}

\textbf{\large Course Objectives:}
Use of language processing tools in Applied Linguistics is increasing day by day especially in Computer Assisted Language Learning (CALL) and Intelligent-CALL (ICALL). Apart from adapting to the technology, learning to tweak software tools you use at work will enable you to customize existing technologies to your needs. This will also let you enhance the technologies with new text processing tools that are derived from your domain expertise. In this context, the objective of this course is to teaching natural language processing methods and tools for applied linguists. In this course, you will learn basic text processing techniques such as - pattern extraction from text, parts of speech tagging, spelling check-correction, text classification, and syntactic parsing. 

\textbf{\large Pre-requisites:}
Knowledge of programming (in any language) is mandatory. 

\bigskip

\textbf{\large Course Details:}
\begin{itemize}
\item  meets in Ross 312, on Tuesdays and Thursdays from 11:00-12:20 pm
\item \textit{Office hours:} Tuesdays and thursdays, 10-11 am (please email beforehand if there are specific issues to discuss. If anyone cannot make it during these hours, send me an email to fix an appointment.)
\end{itemize}

\textbf{\large Credits:} 
\begin{itemize}\vspace*{-.8\baselineskip}\itemsep0ex
\item Credit Points: 3
\\ (Expect to spend more time outside the class to work on the problems and assignments. That is not because I don't know how to teach or you are not smart enough. It is the nature of the course and programming can be frustrating and exciting at the same time even with 20 years of experience.)
\end{itemize}

\bigskip \textbf{Nature of the course and expectations:} This is a 3 credit, graduate level course. Primary mode of instruction is by lectures followed by some discussion. We will have regular assignments, a final project and an oral exam for the project. Readings for each topic are specified in the syllabus and it is expected that the students read them before coming to the class. There may be a few additional (mostly optional) readings or videos from other sources for some of the topics. 

Students enrolled in the course are expected to 
\begin{enumerate}
\item regularly and actively participate in class, and submit the assignments on time (80\% of the grade)
\item finish a programming project as a final exam for the course and attend an oral examination about it (20\% of the grade)
\item work hard, and prepare well for the classes
\end{enumerate}

\bigskip\textbf{\large Grading Policy}
There are 4 assignments with 15 marks each, one assignment for 20 marks and one final project for 20 marks. The assignments will come about once in 2 weeks you usually will have 2 weeks of time for submission. For the final exam, you have to implement a small programming project, which is for 20 marks. The project should be decided by the end of October from a list of given projects. You are allowed to come up with your own idea, but should get my approval before starting to work on that. The final exam grade will also have a short oral defense where you have to explain what you did and answer questions. Plus/minus grading will be used (93\% = A, 90\% = A-, 87\% = B+, 83\% = B, 80\% = B-, etc.). 

\bigskip\textbf{\large Class etiquette:} Please do not read or work on materials for other classes in this class. Come to class on time and
do not pack up early. Electronic devices like mobile phones, tablets etc should not be used in the class. Laptops should be used only for activities related to the classwork. If for some reason, you must leave early or you have an important call coming in, or you have to miss class for an important reason, please let me know (via email) and get it approved \emph{before} the class. Being absent from the class does not allow you to skip submitting any assignments that were assigned in that class. Do not ask questions that can easily be answered by looking at the textbook or in online discussion forums. Value my time, your time, and everyone else's time. Asking such questions will not result in any answers from me \textbf{in the class} even if you write negative feedback about it.  

\bigskip\textbf{\large Academic Conduct}: Generally, you are encouraged to work in groups, discuss, and exchange ideas. At the same time, you are expected to do your assignments by yourself and with honesty. For the text you write, you always have to provide explicit references for
any ideas or passages you reuse from somewhere else. Note that this includes text taken from the web. You should cite the url of the web site in case no more official publication is available. It is common to search in websites such as stackoverflow.com for solutions to some problems or to fix the bugs in your program. However, copying full code samples from somewhere else (including your colleague's program) is considered academic dishonesty. Generally speaking, the class will follow Iowa State University’s policy on academic dishonesty. Anyone suspected of academic dishonesty will be reported to the Dean of Students Office. 

\bigskip\textbf{\large Disability Accommodation}
Iowa State University complies with the Americans with Disabilities Act and Sect 504 of the Rehabilitation Act. If you have a disability and anticipate needing accommodations in this course, please contact (instructor name) to set up a meeting within the first two weeks of the semester or as soon as you become aware of your need.  Before meeting with (instructor name), you will need to obtain a SAAR form with recommendations for accommodations from the Student Disability Resources, located in Room 1076 on the main floor of the Student Services Building. Their telephone number is 515-294-7220 or email disabilityresources@iastate.edu .  Retroactive requests for accommodations will not be honored.

\bigskip\textbf{\large Harassment and Discrimination}
Iowa State University strives to maintain our campus as a place of work and study for faculty, staff, and students that is free of all forms of prohibited discrimination and harassment based upon race, ethnicity, sex (including sexual assault), pregnancy, color, religion, national origin, physical or mental disability, age, marital status, sexual orientation, gender identity, genetic information, or status as a U.S. veteran. Any student who has concerns about such behavior should contact his/her instructor, Student Assistance at 515-294-1020 or email dso-sas@iastate.edu, or the Office of Equal Opportunity and Compliance at 515-294-7612.

\bigskip\textbf{\large Dead Week Policy}
This class follows the Iowa State University Dead Week policy as noted in section 10.6.4 of the Faculty Handbook: \url{http://www.provost.iastate.edu/resources/faculty-handbook}

\bigskip\textbf{\large Textbooks}
\begin{enumerate}
\item Speech and Language Processing - Jurafsky and Martin (I don't insist on buying as it is very expensive, but it is a useful book, and language agnostic). I will be using material from 2nd and 3rd editions. Default is 2nd edition (print book). Lot of draft chapters from this book are freely available online.
\item NLTK Book by Bird, Klein, and Loper (\url{http://www.nltk.org/book/} - free ebook)
\item Language and Computers (Dickinson, Brew and Meurers) - optional. Good to get a general overview, without getting too technical. Slides for this book are available online.
\end{enumerate}
(If you can only buy one textbook, I would suggest "Language and Computers" for non-expert programmers and "Speech and Language Processing" for those who want to continue working in NLP.)

\bigskip\textbf{\large Syllabus - topics covered}

\begin{enumerate}
\item Introduction 
\begin{itemize}
\item Course overview
\item Introduction to Natural Language Processing (NLP)
\item Programming concepts review and practice
\end{itemize}
Readings: Chapter 1 in Jurafsky and Martin, Chapter 1 in NLTK Book - Perl users can also read the chapter to get a general picture. 
%(Problem sets on basics of programming: 5 problems. Ungraded)

\item Text processing
\begin{itemize}
\item Basic text processing - getting word/ngram frequencies, regular expressions, tokenizing and sentence splitting.  
\item Calculating distance between words - application for spelling check-correction.
\item Installing NLP tools for Python and Perl: 
\begin{enumerate}
\item Python: NLTK \url{http://www.nltk.org/install.html} (Install NLTK and NLTK-Data)
\item Perl: Clairlib, Stanford CoreNLP toolset Perl interface \url{http://search.cpan.org/~kal/Lingua-StanfordCoreNLP-0.02/lib/Lingua/StanfordCoreNLP.pm}
%http://orsoftware.blogspot.com/2012/03/text-mining-with-perl-resources.html
\end{enumerate}
\end{itemize}
Readings: Chapter 2 and 3 in J\&M (2nd Edition). Chapter 3 in the NLTK Book (Perl users - I still suggest you to just browse through the chapter)
\\ (Assignment 1 on the first two topics. Two ungraded problem sets with 10 questions each are given for programming practice.) 
%(Problem set on text processing: 10 problems. Ungraded)

\item Morphological analysis
\begin{itemize}
\item Regular expressions
\item Inflectional and Derivational Morphology
\end{itemize}
Readings: Chapters 2,3 in J\&M (2nd Edition). Link to Draft Chapter 2 from J\&M 3rd edition: \url{https://web.stanford.edu/~jurafsky/slp3/2.pdf}
\\ (Assignment 2 on these topics. Ungraded problem set with 10 questions will be given for programming practice.) 
%(Problem set on this part. More on morphological analysis. Regex. FSTs etc: 10 problems. Ungraded)

\item Basics of probability, Language Modeling, Part of speech tagging
\\ Readings: Chapters 4 and 5 in J\&M, Chapter 5 in NLTK book. 
\\ Draft Chapters 4 (\url{https://web.stanford.edu/~jurafsky/slp3/4.pdf}) and 9 (\url{https://web.stanford.edu/~jurafsky/slp3/9.pdf}) of J\&M 3rd Edition has most of this covered.
\\ Lecture Videos: Week 6 and 7 in Radev's coursera course
\\ (Assignment 3 on these topics) 
%(Problem set on this part. Taggers. Using them in code.)

\item Text Classification (Naive bayes, K-nearest neighbours algorithm, logistic regression)
\\  Readings: Chapter 7 in J\&M (Edition 3, url: \url{https://web.stanford.edu/~jurafsky/slp3/7.pdf}), Chapter 6 in NLTK Book. Chapter 5 in Dickinson et.al. book. 
\\ (Assignment 4 on this topic) 
%(Problem set on this part.)

\item Parsing: constituency and dependency parsing
\\ Optional readings: Chapters 12, 13, 14 in J\&M (Watching relevant video lectures will be sufficient).
\\ Videos: Week 4 and 5 in Radev's coursera course
\\ (Assignment 5 on parsing)
%(Problem set on this part.)

\item Overview of other topics in NLP: semantic analysis, discourse analysis; overview of NLP applications (Machine Translation, Information Extraction etc.,)
\\ Optional readings: Chapters 18, 20, 21 in J\&M. 
\\ Optional videos: Week  8--12 in Radev's coursera course.
%(Problem set on this part.)

\item NLP Applications in CALL: writers aids, language tutoring systems
\\ Optional Readings: Chapters 2 and 3 in Dickinson et.al. textbook (going through the slides for these chapters on the book's website will also be fine)
%(Problem set on this part. covers alignment, edit distance, spell check etc.)

\end{enumerate}

\bigskip\textbf{\large Scheduling and Deadlines (tentative)}
Note that the following session plan is subject to change; it only
constitutes the current state of our planning as the semester unfolds.

 \begin{enumerate}\itemsep0ex
%Week 1:
\item Tuesday, August 23: Course orientation; What is NLP? 
%should add readings, etc. course policies awareness form.
%pre-course survey - but on paper. 
%Talk about NACLO website.
%tell them to get their laptops or setup stuff on lab machines.
%Mention also about Radev's course.
%Last 3 lectures from Week 3 in Radev's course. 

\item Thursday, August 25: Why is NLP hard? What are some common text processing problems? Linguistics overview. 
\\ (Assignment 1 given on basic text processing and programming, for 15 marks. Deadline - 10th Sep.) 
%I think I can give a homework to listen to first week lectures of Radev's class: 1.5 hours of lectures.
%Show Jurafsky's first video
%start with a discussion on Radev's lectures from Week 1.
%in class: come up with 5 applications of NLP in CALL
%exercise: word counting in some literature texts etc, ngrams.

%Week 2:
\item Tuesday, August 30: Programming review + Practice 
\\ (Problem Set 1 given for practice in and out of class)
%Programming overview. Ask some algorithms. Drawing flowcharts.
%Ask to write some small programs about texts in the class.

\item Thursday, September 1: Programming Review continued. Installation of NLTK for Python users. Installing Perl libraries (Stanford CoreNLP package) for PERL users. Practice exercises.
%for Python users: ask them to try out some of the examples in Chapter 1 and 2 of NLTK
%for Perl users: come up with equivalent exercises.

%Week 3:
\item Tuesday, September 6:  Regular expressions review and practice.
%show Manning's video (Week1-3) on how they use Regular Expressions in Stanford parser.
%Make them do a 15 min regex inclass exercise - on tokenizing and sentence splitting

\item Thursday, September 8: NLP - Preprocessing tasks (tokenizing, sentence splitting) 
%NLP tasks overview: Radev Week 2
%in class: 20min - come up with an idea to do spelling correction.
%Also: discussion on problem set 1.

%Week 4: 
\item Tuesday, September 13: Preprocessing tasks continued: spelling correction and normalization; Assignment 1 discussion. 
(Assignment 2: Text processing and Regular expression programs. 15 marks. Deadline - 27th Sep. Problem Set 2 given for practice) 
%More on regex: Language description
%In class exercise: stemming?
%spelling correction, other kinds of normalizations.

\item Thursday, September 15:  Morphological Analysis (Book: Chapter 2 and 3 in J\&M)
%Stemming included.
%in class exercise: NACLO morphology problem. 
%http://nacloweb.org/resources/problems/2012/N2012-D.pdf
%look for the solution. Ask about what rules did they use. Ask how they will write a program. 
%30min exercise. 
%\\ Optional video lectures: Radev's Coursera course, Week 2

%Week 5:
\item Tuesday, September 20: Morphological Analysis (Chapter 3 in J\&M + Chap 4 in NLTK book) 
%http://www.nacloweb.org/resources/problems/2014/N2014-A.pdf - Estonian time problem.

\item Thursday, September 22: Ngrams, Language models, POS tagging (Chapter 4 and 5 in J \&M, Chapter 5 in NLTK book)
\\ (Problem set 3 given for practice)
%Prepare training data for tagger experiments!!!
%problem set 3 released
%Lectures before next class: Week 6 and 7 of Radev

%Week 6:
\item Tuesday, September 27: continued. \\
(Assignment 3: on Tagging. 20 Marks, Deadline: 15 October)

\item Thursday, September 29: continued; Assignment 2 discussion. 
%Discussion on Assignment 2.

%Week 7:
\item Tuesday, October 4: continued.
\\ problem set 4 given for practice.

\item Thursday, October 6: Release of example projects for final projects (Deadline to decide: 5 November) and Revision of morphology, and tagging. 

%Week 8: 
\item Tuesday, October 11: Text Classification (Chapter 6 in NLTK)
%Recommend Andrew Ng's lectures

\item Thursday, October 13: Text classification - continued

\item Tuesday, October 18: Text classification - continued. Assignment 3 discussion.\\ (Assignment 4: on text classification and others. 15 marks. Deadline: 29th October)
\\ problem set 5 given for practice

\item Thursday, October 20: Text classification - conclusion. Hui-Hsien's talk on her thesis. General Revision, Midterm feedback
%Decision on Final projects + 1 page reports, Revision and Discussion. Midterm feedback.
%In this class: ask for mid term feedback. Plus Delta thing Volker mentioned.
%Lectures before next class: Radev, Week 4

\item Friday, October 21: Tutorial session @312. 

\item Tuesday, October 25: Parsing, CFGs, PST. 

\item Thursday, October 27: CFGs/PST continued. 

\item Tuesday, November 1: Dependency parsing. 
(Assignment 5: on parsing and others. 15 marks. Deadline: 15th November; Discussion on Assignment 4; problem set 6 released)

\item Thursday, November 3: Parsing - conclusion + Discussion about decisions on Final projects. First report due on 5th November. 

\item Tuesday, November 8:  Other topics in NLP: Discourse analysis, Semantics etc.
%Analyzing the Meaning of Sentences (Chapter 10 in NLTK)

\item Thursday, November 10: NLP applications: Machine Translation, Question Answering etc.
%Managing Linguistic Data (Chapter 11 in NLTK)
\\ problem set 7 for practice. 

\item Tuesday, November 15: NLP and its application to Language learning and assessment, feature extraction from text, text classification etc

\item Thursday, November 17: status report on projects in the class. Discussion on Assignment 5. Problem set 8 released.
%Make this the day to ask about the final project status: May be have one-one meetings with everyone in class. Student conferences?
%(Also announce about make up assignments and criteria) %a max of 10\% extra credit.)
\\ problem set 8 for practice.

\item Tuesday, November 22: Thanksgiving break, no classes

\item Thursday, November 24: Thanksgiving break, no classes

\item Tuesday, November 29: NLP for CALL continued 

\item Thursday, December 1: 

\item Tuesday, December 6: Revision (Last week of classes) 

\item Thursday, December 8: Revision (Last week of classes)
Project submission (20 Marks) - Deadline Friday, December 9 Midnight.

\item December 12-16: Exams week. Oral exams scheduling. 

\end{enumerate}
\end{document}

Final project 25 marks split: 10 marks for reports and oral exam (5 marks for two reports. 5 marks for oral exam. 15 marks for the work)
