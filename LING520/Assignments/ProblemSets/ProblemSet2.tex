\documentclass[11pt,a4paper]{article}
\usepackage{url}

\begin{document}
\begin{center}
  Fall Semester 2016 \\ Iowa State University\\[3ex]
  {\large ENGL 520 - Computational Analysis of English}\\[3ex]
  \textbf{Problem Set 2} \\ \textbf{Basic Text Processing Practice \\ (ungraded)}
\end{center}

%10 questions.
\begin{enumerate}
\item Compare the works of two of your favorite writers in terms of the most frequently used words in their novels, using a computer program. Use the free ebooks available on project gutenberg website for getting the corpus you want.
\item Write a program that converts a user entered sentence into Morse code. Google for morse code to know the mapping between English characters and Morse code sequences.
\item Build a small English to German dictionary file (collect about 100-150 words from the web and save in a .txt file), and write a program that takes a English sentence from user as input, tokenizes it, and looks word by word in the dictionary, and translates the sentence to German. If the word is not found in the dictionary, leave it as it is. If it is found, replace with the translation.
\item Write a program (using regular expressions or any HTML parser tool you know of) that takes a Wikipedia URL as input and returns the number of languages in which the page exists as output. The languages in which the same wiki page exists can be seen in the left frame, at the very end, in all wikipedia pages.
\item Implement a Caesar cipher encoder. Caesar cipher is a simple encryption technique named after Julius Caesar, who used it to communicate with his team in cryptic messages. Usually, Ceasar cipher refers to any encryption where each letter is replaced by another letter a fixed number of places down in the alphabet. For example, if the fixed number is 3, A becomes D, B becomes E and so on. Now, your program should implement a Ceasar cipher of the fixed size 4. Leave the punctuation markers unchanged. Your input will be a normal English sentence, and the output should be a Caesar cipher.
\item Implement a Caeser decoder for the above cipher. Your input is a Ceaser cipher encoded sentence, and your program should decode it and return the decrypted English sentence.
\item Implement the Hangman program as described in this url: \url{http://openbookproject.net/pybiblio/practice/wilson/hangman.php}
\item Write a program that generates a jumbled sentence and asks the user to enter the correctly ordered sentence. If the user enters correctly, inform them that they are right. Otherwise, ask them to enter the correct text. Repeat for a maximum of 5 times. For the generated sentence: Have 10 sentences stored somewhere, and randomly choose one of them, and jumble the order of words before presenting to user.
\item Do problem 46 in \url{http://www.ling.gu.se/~lager/python_exercises.html}. Although the link calls it python exercises, you can solve it in any language you want.
\item Write a program that can read a microsoft word document, and print out the text in it. Refer to the website \url{https://automatetheboringstuff.com} for some relevant Python libraries. Perl users - search for Text::Extract::Word or some such relevant perl module.
\end{enumerate}

\end{document}
