\documentclass[11pt,a4paper]{article}
\usepackage{url}

\begin{document}
\begin{center}
  Fall Semester 2016 \\ Iowa State University\\[3ex]
  {\large ENGL 520 - Computational Analysis of English}\\[3ex]
  \textbf{Problem Set 1} \\ \textbf{General Programming Practice \\ (ungraded)}
\end{center}

%10 questions.
Avoiding to use built in functions as much as possible, write your own implementations for the following:
\begin{enumerate}
\item A program that takes a string as input, and returns its length as output. E.g., python should return 6.
\item A program that takes a character and returns if it is a vowel or a consonant. Return "None" if anything else is entered as input.
\item A program that takes a positive number as input and returns the sum of all integers upto that number as output. So, 4 returns 10 (1+2+3+4) as output.
\item A program that takes a string as input, and returns the output as true if the string is a palindrome, and false if it is not.
\item A program that takes two numbers (without decimals) as input and returns the number of digits in common between them. For example, if your inputs are: 1234 and 235, your program should return 2 (since two numbers: 2 and 3 occur in both numbers)
\item A program that given a text file will create a new text file in which all the lines from the original file are numbered from 1 to n (where n is the number of lines in the file). source: \url{http://www.ling.gu.se/~lager/python_exercises.html} 
\item A program that takes a text file as input, checks for lines ending in ? and returns the number of such lines as output.
\item A program that prints today's date, week day, current time and current week's number in the year.
\item A program to multiply two numbers without using the * sign/operation.
\item A program that takes 3 numbers as input and checks if they form a Pythagorean triplet. Google the term if you do not know what it is!
\end{enumerate}

\end{document}
